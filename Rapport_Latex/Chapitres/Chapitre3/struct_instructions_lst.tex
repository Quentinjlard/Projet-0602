\lstset{style=mystyle}
\begin{lstlisting}[language=C, caption=Structure pour une liste instruction]
typedef struct instruction_list_node_type {
    instruction_t* instruction;
    struct instruction_list_node_type* next;
} instruction_node_t;

typedef struct instruction_list_type{
    instruction_node_t* head;
    instruction_node_t* tail;
} instruction_list_t;

typedef struct call_procedure_data_type{
    char* procedure_name;
    instruction_list_t* arguments;
} call_procedure_data_t;

typedef struct assignment_data_type{
    char* variable_name;
    instruction_t* value_expression;
} assignment_data_t;

typedef struct conditional_data_type{
    instruction_t* condition_expression;
    instruction_list_t* then_instructions;
    instruction_list_t* else_instructions;
} conditional_data_t;

typedef struct while_loop_data_type{
    instruction_t* condition_expression;
    instruction_list_t* loop_instructions;
} while_loop_data_t;

typedef struct for_loop_data_type{
    symbol_t* variable;
    symbol_t* start;
    symbol_t* end;
    int step;
    instruction_list_t* loop_instructions;
} for_loop_data_t;
\end{lstlisting}

La structure \textit{instruction\_node\_t} contient un pointeur vers une isntruction et un pointeur vers le noeud suivant.
 
La structure \textit{instruction\_list\_t} contient un pointeur de la liste d'instruction et un pointeur vers la tête de la liste.

Pour les structures \textit{call\_procedure\_data\_t}, \textit{assignment\_data\_t}, \textit{conditional\_data\_t}, \textit{while\_loop\_data\_t} et \textit{for\_loop\_data\_t} contienent les données nécessaires à la bonne exécution du code.

\newpage