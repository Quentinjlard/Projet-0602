

%"mystyle" code listing set
\lstset{style=mystyle}
\begin{lstlisting}[language=C, caption=Structure d'uun Block]
typedef enum block_name_type{
    EMPTY,
    BLOCK,
    TRAP,
    LIFE,
    BOMB,
    DOOR,
    ENTER,
    EXIT,
    LADDER,
    ROBOT,
    PROBE,
    KEY,
    GATE
} block_name_t;

typedef struct block_type{
    block_name_t type;
    int value;
    int coordX;
    int coordY;
} block_t;
\end{lstlisting}

Les structures présentent ci-dessus sont une énumération nommée \textit{block\_name\_type} et une structure nommée \textit{block\_t}.

L'énumération \textit{block\_name\_type} définit différents types de blocs que nous pouvons reprouvé dans un niveau du jeu.

la structure \textit{block\_t} contient quatre membre. Le premier est son type du block présent sur la map de type \textit{block\_name\_type}. Le second est la value par défaut la value est à 0 néamois quand il s'agit d'une porte, d'une clé sa valeur change, la value est de de type \textit{int}. Un bloc posséde une position sur notre map en X et Y pour cela nous avons deux variable intitulé \textit{coordX} et \textit{coordY} qui sont de type \textit{int}.

Pour conclure, la structure \textit{block\_t} permet de stocker les informations de chaque blocs présent sur la map du jeu. 

\newpage