%"mystyle" code listing set
\lstset{style=mystyle}
\begin{lstlisting}[language=C, caption=Structure d'un level]
// Grid size
#define HEIGHT     20
#define WIDTH      60

// Structure of a level
typedef struct {
    wint_t cells[HEIGHT][WIDTH];
    color_t colors[HEIGHT][WIDTH];
    block_t blocks[HEIGHT][WIDTH];
} level_t;
\end{lstlisting}

La strcture présente ci-dessus se nomme \textit{level\_t}, elle est utilisé dans notre projet pour gérer des niveau dans un jeu.

Cette structure est composée de trois tableaux bidimensionnels nommés "cells", "colors" et "blocks" de dimensions "HEIGHT" par "WIDTH".

Le tableau \textit{cells} est composé d'élément de type \textit{wint\_t} qui représente des cellule présente sur la grille du niveau. Le tableau \textit{colors} cotient des éléments de type \textit{color\_t} qui stocke la couleur de chaque celulle. Le tableau \textit{blocks} est composé d'élement de type \textit{block\_t} qui stockent les blocks présents dans chaque \textit{cells}.


Nous pouvons observer la présence de deux variables globales intitulé \textit{HEIGHT} et \textit{WIDTH} qui représente respectivement la hauteur et la largeur de la grille.


En conclusion, cette structure nous permet de stocker les informations nécessaire pour réprésenter un niveau de jeu pour chaque cellule de notre niveau.
\newpage