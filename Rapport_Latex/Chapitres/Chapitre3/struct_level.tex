%"mystyle" code listing set
\lstset{style=mystyle}
\begin{lstlisting}[language=C, caption=Structure d'un level]
// Grid size
#define HEIGHT     20
#define WIDTH      60

// Structure of a level
typedef struct {
    wint_t cells[HEIGHT][WIDTH];
    color_t colors[HEIGHT][WIDTH];
    block_t blocks[HEIGHT][WIDTH];
} level_t;
\end{lstlisting}

La structure présente ci-dessus se nomme \textit{level\_t}, elle est utilisée dans notre projet pour gérer des niveaux dans un jeu.

Cette structure est composée de trois tableaux bidimensionnels nommés "cells", "colors" et "blocks" de dimensions "HEIGHT" par "WIDTH".

Le tableau \textit{cells} est composé d'éléments de type \textit{wint\_t} qui représente des cellule présente sur la grille du niveau. Le tableau \textit{colors} contient des éléments de type \textit{color\_t} qui stocke la couleur de chaque cellule. Le tableau \textit{blocks} est composé d'élement de type \textit{block\_t} qui stockent les blocks présents dans chaque \textit{cells}.


Nous pouvons observer la présence de deux variables globales intitulées \textit{HEIGHT} et \textit{WIDTH} qui représentent respectivement la hauteur et la largeur de la grille.


En conclusion, cette structure nous permet de stocker les informations nécessaires pour réprésenter un niveau de jeu pour chaque cellule de notre niveau.
\newpage