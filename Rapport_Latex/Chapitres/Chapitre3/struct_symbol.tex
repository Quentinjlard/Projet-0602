%"mystyle" code listing set
\lstset{style=mystyle}
\begin{lstlisting}[language=C, caption=Structure d'un symbole]
typedef struct symbol_type {
    char* name;
    int value;
} symbol_t;
\end{lstlisting}

Dans le code ci-dessus nommée \textit{symbol\_t}, nous avons deux membres. Le premier membre est un pointeur vers un table de cractéres qui défini le nom du symbole. Le second membre est un entier qui correspond à la valeur d'affectation du symbole, il est de type int.

Néanmois, il faut noter que le nom d'un symbole n'a pas de taille fixe vue qu'il s'agit d'un pointeur vers un char.

\newpage