\lstset{style=mystyle}
\begin{lstlisting}[language=C, caption=Structure pour une instruction]
typedef enum instruction_type_type{
    CALL_PROCEDURE_INSTRUCTION,
    ASSIGNMENT_INSTRUCTION,
    CONDITIONAL_INSTRUCTION,
    WHILE_LOOP_INSTRUCTION,
    FOR_LOOP_INSTRUCTION,
    LADDER_INSTRUCTION,
    RECT_INSTRUCTION,
    FRECT_INSTRUCTION,
    HLINE_INSTRUCTION,
    VLINE_INSTRUCTION,
    GATE_INSTRUCTION,
    PUT_INSTRUCTION
} instruction_type_t;

typedef struct ladder_data_type{
    symbol_t* x;
    symbol_t* y;
    symbol_t* h;
} ladder_data_t;

typedef struct rect_data_type{
    symbol_t* x1;
    symbol_t* y1;
    symbol_t* x2;
    symbol_t* y2;
    char* block;
} rect_data_t;

typedef struct frect_data_type{
    symbol_t* x1;
    symbol_t* y1;
    symbol_t* x2;
    symbol_t* y2;
    char* block;
} frect_data_t;

typedef struct hline_data_type{
    symbol_t* x;
    symbol_t* y;
    symbol_t* l;
    char* block;
} hline_data_t;

typedef struct vline_data_type{
    symbol_t* x;
    symbol_t* y;
    symbol_t* l;
    char* block;
} vline_data_t;

typedef struct gate_data_type{
    symbol_t*  x;
    symbol_t* y;
    symbol_t* n;
} gate_data_t;

typedef struct put_data_type{
    symbol_t* x;
    symbol_t* y;
    char* block;
} put_data_t;

typedef struct instruction_type{
    instruction_type_t type;
    void* data;
} instruction_t;
\end{lstlisting}

Dans ce code, nous avons l'ensemble des structures pour les instructions que nous avons besoin pour le projet.

Chaque structure de données est définie avec des champs spécifique pour stocker les informations qu'ils ont besoin telles que les coordonnées en X et Y, la hauteur etc...

\newpage