%"mystyle" code listing set
\lstset{style=mystyle}
\begin{lstlisting}[language=C, caption=Structure d'une table de symbole]
typedef struct table_type {
    lst_symbol_t* list;
    lst_symbol_t* head;
} table_t;
\end{lstlisting}

La structure \textit{table\_t} est une structure de données qui contient deux membres. Le premier membre est un pointeur de type \textit{lst\_symbol\_t} qui pointe vers une liste chaînée de symboles. Le second champ est de même type que le premier et il pointe vers la tête de la liste.

La liste chaînée de symbole est stockée dans l'élément \textit{list}. Elle est parcourue par l'utilisation du pointeur suivant. 
Le pointeur \textit{head} pointe vers le 1er élément de la liste alors que le dernier élement de la liste à pour pointeur \textit{next} la valeur NULL.
\newpage