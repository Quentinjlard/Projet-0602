\lstset{style=mystyle}
\begin{lstlisting}[caption=Union Yacc]
%union {
        block_t block;
        int value;
        int coordX;
        int coordY;
        char* lettre;
        symbol_t* symbol;
    }

\end{lstlisting}

Le bloc si-dessus est un bloc d'union qui est utilisé pour stocker divers types de données durant l'analyse syntaxique. Nous utiliserons dans notre projet six type de données différentes : 

\begin{itemize}
    \item block\_t block : Membre de type block\_t vue dans le chapitre précédant.
    \item int value : Champ de type int qui se nomme value dans notre code Yacc
    \item int coordX: Champ pour stocker une coordonnée X
    \item int coordY: Champ pour stocker une coordonnée Y
    \item char* lettre: Champ de type char* pour stocker des mots ou des lettres
    \item symbol\_t* symbol: Champs permettant le stockage de tous type de symboles.
\end{itemize}

\newpage