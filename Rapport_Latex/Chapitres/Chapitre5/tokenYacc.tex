\lstset{style=mystyle}
\begin{lstlisting}[caption=Token Yacc]
    %token LEVEL END

    %token EMPTY_YACC BLOCK_YACC TRAP_YACC LIFE_YACC BOMB_YACC DOOR_YACC ENTER_YACC EXIT_YACC LADDER_YACC ROBOT_YACC PROBE_YACC KEY_YACC GATE_YACC
    %token BLOCK_VAL_YACC

    %token GET PUT


    %token PARO PARF VIRG NUM PVRIGULE
    %token SUP

    %token ADDITION SOUSTRACTION DIVISION MULTIPLICATION EGALE SUPEGAL

    %token SYMBOLE

    %token PRC_YACC LADDER_PRC_YACC RECT_YACC FRECT_YACC HLINE_YACC VLINE_YACC 

    %token IF_YACC THEN_YACC ELSE_YACC

    %token WHILE_YACC DO_YACC FOR_YACC TO_YACC STEP_YACC

\end{lstlisting}

En Yacc, un token est une unité lexical. Dans le code ci-dessus, vous trouverez l'ensemble de nos tokens utilisés dans notre fichier Yacc.
Les tokens sont un lien entre le fichier Lex et le fichier Yacc.

\newpage