\lstset{style=mystyle}
\begin{lstlisting}[caption=Régle généraliste pour la bonne lecture d'un fichier]
file: level_file_list
        | instruction_proc_list
        ;

    level_file_list: level_file
                    | level_file_list level_file
                ;

    instruction_proc_list: instruction_proc
                    | instruction_proc_list instruction_proc
                    ;

    instruction_list :  instruction_list instruction
                |   instruction
                ;

    instruction :   instructionPUTNombre
                |   instructionPUTVariable
                |   instructionProcedure
                |   affectation
                |   END {....}
                ;
\end{lstlisting}

La règle de grammaire nommé \textit{file} possède deux options. La première option a pour but de contruire une liste de fichiers de niveau.  Ellle permet de combiner plusieurs niveaux dans un fichier. La seconde option permet de contruire une liste d'intructions procédure.

La règle de grammaire nommée \textit{level\_file\_list} possède deux options également. La premier est dans le cas où nous avons qu'un seul niveau dans notre fichier. La seconde est écrite dans le cas ou nous avons plusieurs niveaux dans le même fichier.

La règle de grammaire nommée \textit{instruction\_proc\_list} possèdent deux options aussi pour les même raisons que la grammaire \textit{level\_file\_list}.

La règle de grammaire \textit{isntruction} possèdent de multiples possibilitées de travail. La première consiste a faire exécuter des option PUT avec des coordonnés X et Y qui sont numériques. La seconde option consiste à remplacer les coordonnés X et Y par des variables ( symbole ) qui peuvent subire des opérations de calcul entre elles. La troisième sert dans le cas où nous devons poser un bloc sur la map depuis une instruction/procédure définie auparavant. la troisième est une opération d'affection, elle permet d'assigner des valeurs à des variables (symbole). La dernière règle est le TOKEN end pour dire qu'il s'agit de la fin d'un cycle et peut-être la fin du fichier. Les \textit{\{...\}} signifie qu'il y a des instructions en C dans cette accolade qu'on ne developpera pas dans le rapport.
\newpage

\lstset{style=mystyle}
\begin{lstlisting}[caption=Lecture d'un niveau]
    level_file: LEVEL {....} instruction_list ;
\end{lstlisting}

La règle \textit{level\_file} commence par le token LEVEL ce qui marque le début d'un niveau. Nous avons ensuite une suite d'instruction en C. Pouis finir nous faisons appel à une autre grammaire présente  plus haute pour pouvoir créer des cycles de lecture du fichier.

\lstset{style=mystyle}
\begin{lstlisting}[caption=Grammaire pour l'instruction d'une procédure]
    instruction_proc : 
                    
                        PRC_YACC
                        {
                            ...
                        }    LADDER_PROC FOR_LOOP_PROC PUT_PROC END
                    |   PRC_YACC{
                            ...
                        }   RECT_PROC FOR_LOOP_PROC PUT_PROC PUT_PROC END FOR_LOOP_PROC PUT_PROC PUT_PROC END
                    |   PRC_YACC{
                            ...
                        }   FRECT_PROC FOR_LOOP_PROC FOR_LOOP_PROC PUT_PROC END END
                    |   PRC_YACC{
                            ...
                        }   HLINE_PROC FOR_LOOP_PROC PUT_PROC END   
                    |   PRC_YACC{
                            ...
                        }   VLINE_PROC FOR_LOOP_PROC PUT_PROC END
                    |   END
                    {
                        ...
                    }
                    | level_file
                    ;
    ;
\end{lstlisting}

Nous allons voir à present la règle de production qui se nomme \textit{instructions\_proc}.
Cette règle est composé de sept membres. Nous allons voir pour commencer les cinqs premiers puis nous verrons les deux dernières par la suite.

Les cinq premières instructions commencent par un symbole non-terminal (PRC\_YACC) suivi d'une séquence d'autre symboles non-terminaux et terminaux.
Avec les insctructions procédures (PRC\_YACC) données dans l'extrait de code ci-dessus, nous pouvons réaliser les procédures donné dans le fichier texte test.
Si nous prenons pour exemple la procédure pour poser une échelle, nous pouvons voir que nous aurons une boucle dans laquelle nous auront une instrcution PUT qui nous permettera de la poser sur la map. A la suite de cela nous avons le 'END' qui permet de donner la fin des instruction présente dans la boucles.
Nous avons garder cette même réflexion pour les autres procédures.

L'instruction END qui est à la fin nous permet de donner la fin d'une procédure.
L'instruction \textit{level\_file}, nous permet de passer sur la génération d'un monde à la suite d'un enregistrement de divers procédures dans notre mémoire.

\lstset{style=mystyle}
\begin{lstlisting}[caption=Régle de grammaire pour les différentes procédures]
LADDER_PROC : LADDER_PRC_YACC PARO affectation VIRG affectation VIRG affectation PARF 
                {
                   ...
                }
                ;

    RECT_PROC : RECT_YACC PARO affectation VIRG affectation VIRG affectation VIRG affectation VIRG affectation PARF 
                {
                    ...
                };

    FRECT_PROC : FRECT_YACC PARO affectation VIRG affectation VIRG affectation VIRG affectation VIRG affectation PARF 
                {
                    ...
                };

    HLINE_PROC : HLINE_YACC PARO affectation VIRG affectation VIRG affectation VIRG affectation PARF 
                {
                    ...
                };

    VLINE_PROC : VLINE_YACC PARO affectation VIRG affectation VIRG affectation VIRG affectation PARF 
                {
                    ...
                };

    FOR_LOOP_PROC : 
                FOR_YACC PARO affectation PVRIGULE affectation SUPEGAL SYMBOLE PVRIGULE SYMBOLE EGALE SYMBOLE ADDITION NUM PARF
                {
                    ...
                }
                | FOR_YACC PARO affectation PVRIGULE affectation SUPEGAL SYMBOLE ADDITION SYMBOLE SOUSTRACTION NUM PVRIGULE SYMBOLE EGALE SYMBOLE ADDITION NUM PARF
                {
                    ...
                }
                | FOR_YACC PARO affectation PVRIGULE affectation SUP affectation ADDITION affectation PVRIGULE SYMBOLE EGALE SYMBOLE ADDITION NUM PARF
                {
                    ...
                }
                ;

    PUT_PROC : PUT PARO SYMBOLE VIRG SYMBOLE VIRG LADDER_YACC PARF  
                {
                    ...
                }
                |
                PUT PARO SYMBOLE VIRG SYMBOLE VIRG SYMBOLE PARF  
                {
                    ...
                }
                ;

\end{lstlisting}

L'ensemble des instructions ci-dessus représente les procédure que nous pouvons rencontrer dans nos fichiers de test. L'ensemble des procédures prennent en entrée et produisent dans le code {...} un résultat de sortie comme le stockage des varaibles nécessaires à la bonne exécution de l'ensemble des procédures.


Nous allons vous expliquer rapidement chaque instruction : 

\begin{itemize}

\item \textit{LADDER\_PROC} est une instruction qui permet de créer une échelle en utilisant trois paramètres. Les trois paramètres sont générés depuis la grammaire d'affectation que nous aborderons plus tard.

\item \textit{RECT\_PROC} est une instuction qui a pour but de créer un rectangle en utilisant cinq paramètres en utilisant la règle de grammaire d'affectation.

\item \textit{FRECT\_PROC} est une instuction qui a pour but de créer un rectangle plein en utilisant cinq paramètres en utilisant la règle de grammaire d'affectation.

\item \textit{HLINE\_PROC} est une instuction qui a pour but de créer une ligne horizontale en utilisant trois paramètres en utilisant la règle de grammaire d'affectation.

\item \textit{VLINE\_PROC} est une instuction qui a pour but de créer une ligne verticale en utilisant trois paramètres en utilisant la règle de grammaire d'affectation.

\item \textit{FOR\_LOOP\_PROC} est une instruction qui permet de créer une boucle for avec plusieurs symboles qui sont directement gérés par affectation. Il y a trois type de boucle spécifiés dans cette grammaire car nous avons les boucles for avec une condition d'arret superieur ou égale ou juste supérieur et celle où nous faisons des sommes de différents symboles pour obtenir une condition d'arret. Les paramètres nécessaires sont : une variable de départ, une variable de fin et un pas d'augmentation d'où le \textit{ SYMBOLE EGALE SYMBOLE ADDITION NUM} qui permet de définir le pas.

\item \textit{PUT\_PROC} Il s'agit d'une instruction qui met en place à l'écran le bloc demandé. Quand il s'agit d'une échelle, nous avons fait une règle spécifique vis à vis d'un problème rencontré avec la règle dans le fichier Lex sur LADDER.

\end{itemize}

\newpage

\lstset{style=mystyle}
\begin{lstlisting}[caption=]
instructionPUTNombre : 
        PUT PARO expression VIRG  
        {
            ...
        } 
        expression VIRG 
        {
            ...
        }
        instructionBlock PARF
        ;
\end{lstlisting}

Cette règle de grammaire définit la syntaxe pour une instruction "PUT" qui met en place un block sur une map level.
Nous allons voir chaque partie de cette instruction.
Pour la première partie, nous définissions la coordonnée en X de notre bloc. Pour la seconde partie, nous faisons de même mais pour la coordonnée en Y. Dans la dernière ligne de cette instruction, nous allons rechercher les la grammaire instructionBlock qui nous sera détaillée plus tard mais qui permet de placer un block sur la map.

\lstset{style=mystyle}
\begin{lstlisting}[caption=]
instructionPUTVariable :
        PUT PARO SYMBOLE VIRG
        {
            ...
        } expression VIRG 
        {
            ...
        }
        instructionBlock PARF
        |
        PUT PARO expression  VIRG
        {
            ...
        } SYMBOLE VIRG 
        {
            ...
        }
        instructionBlock PARF
        |
        PUT PARO SYMBOLE  VIRG
        {
            ...
        } SYMBOLE VIRG 
        {
            ...
        }
        instructionBlock PARF
        ;   
\end{lstlisting}

La règle de grammaire \textit{instructionPUTVariable} permet de faire passer soit une variable ou un nombre en paramètre pour donner les coordonés d'un bloc. Elle appelera ensuite la règle de grammaire \textit{instructionBlock} en lui passant des coordonnées X et Y. Nous verrons la règle \textit{instructionBlock} dans la suite de ce rapport.

\lstset{style=mystyle}
\begin{lstlisting}[caption=Affectation d'un symbole]
    affectation : 
                SYMBOLE 
                {
                    ...
                }
                |
                SYMBOLE EGALE NUM 
                {
                    ...
                }
                |
                SYMBOLE EGALE SYMBOLE 
                {
                    ...
                }
                |
                SYMBOLE EGALE SYMBOLE ADDITION NUM {
                    ...
                }
                |
                SYMBOLE EGALE SYMBOLE ADDITION SYMBOLE {
                    ...
                }
                ;
\end{lstlisting}

La règle de grammaire d'\textit{affectation} consiste à donner une valeur à une varaible. Dans cette grammaire, nous avons plusieurs règles pour spécifier les types d'affectation possible.

La première règle : il s'agit de transformer une lettre en un symbole en mettant la valeur de la lettre à 0. ( X );

La deuxième règle : a pour but d'affecter un nombre à une variable en utilisant un traitement en langage C adapté ( X = 2 ) ;

La troisième règle : a pour but de faire la somme du contenu d'une  variable avec un nombre ( X = X + 1 );

La quatrième règle : a pour but de faire la somme de deux symboles. ( X = X + Y );

Chaque bloc de règle est suivi par des accolades qui permettent de réaliser les actions à effectuer. Ces actions peuvent être la création d'un symbole dans la table de symbole, la mise à jour d'un symbole en modifiant la variable ou réaliser tout type d'opération nécessaire à la bonne gestion des symboles.


\newpage
\lstset{style=mystyle}
\begin{lstlisting}[caption=Expression de calcul pour les nombres]
    expression : 
        | NUM ADDITION NUM	{ ... }
        | NUM SOUSTRACTION NUM	{ ... }
        | NUM MULTIPLICATION NUM	{ ... }
        | NUM DIVISION NUM	
        {
            ...
        }
        | '(' NUM ')'		{... }
        | NUM { ... }
        | NUM EGALE NUM { .. }
        ;
\end{lstlisting}

La régle de grammaire \textit{expression} permet de réaliser des opérations mathématique entre deux nomnbre.
Les opérations mathématiques que nous pouvons réaliser sont les additions, les soustractions, les multiplication, les divisions ( avec verification des diviseurs ) ainsi que comparée si deux nombres sont égales.

\lstset{style=mystyle}
\begin{lstlisting}[caption=instruction pour un Block]
    instructionBlock : 
                        block
                        {
                            ...
                        }
                        |
                        GET PARO NUM VIRG NUM PARF 
                        {
                            ...
                        }
                        ;
\end{lstlisting}

La règle de grammaire \textit{instructionBlock} permet soit de lancer l'édition d'un bloc sur la map ou soit de montrer le contenu d'un bloc sur un point précis de la map.
La première règle de grammaire envoie directement sur la grammaire bloc qu'on verra par la suite  \textit{block}.
La seconde règle de grammaire a pour but de récupérer les valeurs et de les mettre dans des variables X et Y pour ensuite réaliser une recherche dans notre tableau de Blocs.

\lstset{style=mystyle}
\begin{lstlisting}[caption=Bloc]
block: BLOCK_YACC 
                        {
                            ...
                        }
        | TRAP_YACC 
                        {
                            ...
                        }
        | LIFE_YACC
                        {
                            ...
                        }
        | BOMB_YACC 
                        {
                            ...
                        }
        | DOOR_YACC PARO expression PARF
                        {   
                            ...
                        }
        | ENTER_YACC 
                        {
                            ...
                        }
        | EXIT_YACC 
                        {
                            ...
                        }
        | LADDER_YACC 
                        {
                            ...
                        }
        | ROBOT_YACC 
                        {
                            ...
                        }
        | PROBE_YACC 
                        {
                            ...
                        }
        | KEY_YACC PARO expression PARF
                        {
                            ...
                        }
        | GATE_YACC PARO expression PARF 
                        {
                            ...
                        }
                ;
\end{lstlisting}

Dans cette grammaire, nous pouvons voir que nous avons que le noms des blocks (et pour certains une expression).

Pour tous ceux qui où il y a que le nom du bloc on exécutera directement la méthode pour l'intégrer à la map.
Pour ceux qui possèdent une expression permettent d'ajouter une option tel que le numéro de la porte ou le numéro de la clé / Porte. 


\newpage

\lstset{style=mystyle}
\begin{lstlisting}[caption=Bloc]
    instructionProcedure : 
        FRECT_YACC PARO NUM VIRG NUM VIRG NUM VIRG NUM VIRG BLOCK_YACC PARF
        {
            ...
        } 
        ;    
\end{lstlisting}

La grammaire sert à lire les appels procédure dans la partie LEVEL ... END

\newpage