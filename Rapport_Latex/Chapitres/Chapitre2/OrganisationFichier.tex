Dans notre archive, une fois décompressé, vous trouverez de nombreux fichiers.
Tout d'abord, nous avons fait le choix de ne pas faire de sous dossiers pour simplifier le travail.

Néamois, il y a divers fichiers : 
\begin{itemize}
    \item .c : Des fichiers sources
    \item .h : des fichiers d'en-têtes
    \item .yacc : Fichier de grammaire Yacc
    \item .lex : Fichier d'analyse lexicale

\end{itemize}

Dans l'archive rendu, vous trouverez quatres dossiers.
Le premiere dossier est le Code C qui nous a été fournis. Le second dossier est intitulé \textit{Code\_Lex\_Yacc\_En\_developpement} celui-ci regroupe l'ensemble des codes que nous avons réaliser avec une version fini qui comporte des beugs. Nous pouvons dire qu'il s'agit d'une version beta de nos codes. Le troisiéme dossiers est intitulé \textit{Code\_Lex\_Yacc\_fonctionnel} regroupe l'ensemble de nos codes tester et fonctionnel sur l'ensemble des fichiers test présent dans le quatriéme dossier. Le quatriéme dossier regroupe l'ensemble des fichiers de test utilisés dans notre projet. 