\lstset{style=mystyle}
\begin{lstlisting}[language=Lex, caption=Option Lex du projet]
%option nounput
%option noinput
%option yylineno
\end{lstlisting}

Nous avons utilisé trois options pour ce programme, la première est le nounput, la seconde noinput et la dernière yylineo :

La première option utilisée est le noumput. Cette option permet d'éviter la génération d'erreur lorsque l'analyseur lexical rencontre une entrée invalide.

La seconde option est le noinput. Cette option est utile lorsque l'analyseur lexical ne doit pas lire sur l'entrée standard.

La troisiéme option est le yylieno. Cette option permet d'activer le suiveur de ligne dans l'analyseur lexical. Les numéros de lignes sont stockés dans une variable globale.Nous nous en servirons pour indiquer ou est ce qu'il y a une erreur de lecture dans notre fichier.