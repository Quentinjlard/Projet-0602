\lstset{style=mystyle}
\begin{lstlisting}[language=Lex, caption=Option Lex du projet]
%option nounput
%option noinput
%option yylineno
\end{lstlisting}

Nous avons utilisés trois options pour ce programme, la premiére est le nounput, la seconde noinput et la derniére yylineo. Nous allons vous expliqué qu'elle est le rôle de chaque option.

La premiére option utilisé est le noumput. Cette option permet d'éviter la génération d'erreur lorsque l'analyseur lexical rencontre une entrée invalide.

La seconde option est le noinput. Cette option est utile lorsque l'analyseur lexical ne doit pas lire sur l'entrée standard.

La troisiéme option est le yylieno. Cette option permet d'activer le suivre de ligne dans l'analyseur lexical. Les nuémro de ligne sont stocké dans un variable globale.Nous nous en servirons pour indiqué ou est qu'il y a une erreur de lecture dans notre fichier.